\documentclass[letterpaper, 8pt]{extarticle}
\usepackage{amssymb,amsmath,amsthm,amsfonts}
\usepackage{multicol,multirow}
\usepackage{calc}
\usepackage{ifthen}
\usepackage[landscape]{geometry}
\usepackage[colorlinks=true,citecolor=blue,linkcolor=blue]{hyperref}
\usepackage{booktabs}
\usepackage{ulem}
\usepackage{enumitem}
\usepackage{tabulary}
\usepackage{graphicx}
\usepackage{siunitx}
\usepackage{tikz}
\usepackage{derivative}
\usepackage{svg}
\usepackage{listings}
\usepackage{color}
\usepackage{soul}
\usepackage{clrscode3e}


\ifthenelse{\lengthtest { \paperwidth = 11in}}
    { \geometry{top=.25in,left=.25in,right=.25in,bottom=.3in} }
	{\ifthenelse{ \lengthtest{ \paperwidth = 297mm}}
		{\geometry{top=1cm,left=1cm,right=1cm,bottom=1cm} }
		{\geometry{top=1cm,left=1cm,right=1cm,bottom=1cm} }
	}

\newenvironment{Figure}
  {\par\medskip\noindent\minipage}
  {\endminipage\par\medskip}

\pagestyle{empty}
\makeatletter
\renewcommand{\section}{\@startsection{section}{1}{0mm}%
                                {-1ex plus -.5ex minus -.2ex}%
                                {0.5ex plus .2ex}%x
                                {\normalfont\normalsize\bfseries}}
\renewcommand{\subsection}{\@startsection{subsection}{2}{0mm}%
                                {-1explus -.5ex minus -.2ex}%
                                {0.5ex plus .2ex}%
                                {\normalfont\small\bfseries}}
\renewcommand{\subsubsection}{\@startsection{subsubsection}{3}{0mm}%
                                {-1ex plus -.5ex minus -.2ex}%
                                {1ex plus .2ex}%
                                {\normalfont\tiny\bfseries}}
\makeatother
\setcounter{secnumdepth}{0}
\setlength{\parindent}{0pt}
\setlength{\parskip}{0pt plus 0.5ex}
% -----------------------------------------------------------------------
% \tymin=37pt
% \tymax=\maxdimen

% Custom siunitx defs
\DeclareSIUnit\noop{\relax}

\NewDocumentCommand\prefixvalue{m}{%
\qty[prefix-mode=extract-exponent,print-unity-mantissa=false]{1}{#1\noop}
}

% Shorthand definitions
% \newcommand{\To}{\Rightarrow}

% condense itemize & enumerate
\let\olditemize=\itemize \let\endolditemize=\enditemize \renewenvironment{itemize}{\olditemize \itemsep0em}{\endolditemize}
\let\oldenumerate=\enumerate \let\endoldenumerate=\endenumerate \renewenvironment{enumerate}{\oldenumerate \itemsep0em}{\endoldenumerate}

\title{2GA3}

\begin{document}

\raggedright
\tiny

\begin{center}
	{\textbf{2GA3}} \\
\end{center}
\begin{multicols*}{4}
	\setlength{\premulticols}{1pt}
	\setlength{\postmulticols}{1pt}
	\setlength{\multicolsep}{1pt}
	\setlength{\columnsep}{2pt}

	\section{Logic Basics}
	\subsection{Physics}
	\textbf{Ohm's Law:} $R = \frac{U}{T}$\\
	\textbf{Series:} \\
	$R_T = \sum_{i=1}^N R_i$ \\
	$V_T = \sum_{i=1}^N V_i$ \\
	$I_T = I_1 = I_2 = \dots = I_N$\\
	\textbf{Parallel:} \\
	$1/R = \sum_{i=1}^N (1/R_i)$\\
	$V_T = V_1 = V_2 = \dots = V_N$\\
	$I_T = \sum_{i=1}^N I_i$ \\

	\subsection{Transistors}
	% REVIEW: Do we want an image here?
	MOSFETs have 4 components: Source, Gate, Drain, and Base

	\textbf{PNP/NMOS:} Is on when gate is positive. Does not have circle.\\
	\textbf{NPN/PMOS:} Is on when gate is negative. Has circle.

	Generally, transistors are used to pull the output to either
	a positive voltage, or a zero voltage (1 or 0, on or off).
	If output is not pulled to one of these,
	the output is floating and is indeterminate in voltage.

	\subsection{Logic Circuits}
	\subsubsection{Symbols}

	% REVIEW: No clue if these are readable, needs to be printed
	% and double checked before release
	\begin{center}
		\includegraphics[width=.8\linewidth]{logic-gates.png}
	\end{center}
	\subsubsection{Adders}
	\begin{center}
		\includegraphics[width=.3\linewidth]{half-adder.png}
		\includegraphics[width=.6\linewidth]{full-adder.png}
		Left: Half-adder. Right: Full-adder.
	\end{center}
	% REVIEW: Is there anything else that needs to be added here?
	Half adders have no way to carry input from previous sums.
	To add more bits, just chain multiple full-adders together using the carry-out
	bit.
	If final carry at end is 1, that signifies an overflow error.

	\subsubsection{Latches \& Flip-flops}
	\textbf{Flip-flop}
	Every time the input switches from 0 to 1, the output switches to the opposite.
	% TODO: Add more detail about flip-flops

	\textbf{Gated D-latch}
	\begin{center}
		\includegraphics[width=.6\linewidth]{gated-d-latch.png}
		\begin{tabular}[!ht]{@{}cc|ccc@{}}
			\toprule
			$E/C$ & $D$ & $Q$               & $\overline{Q}$               & Comment   \\
			\midrule
			0     & X   & $Q_\textit{prev}$ & $\overline{Q}_\textit{prev}$ & No change \\
			1     & 0   & 0                 & 1                            & Reset     \\
			1     & 1   & 1                 & 0                            & Set       \\
			\bottomrule
		\end{tabular}
	\end{center}
	Operate on the principle of propagation delay.

	% REVIEW: Do we need this?
	% seems like a lot of space dedicated to something that's trivial to derive
	% from scratch
	Stacking many of them can be used to create a register:
	\includegraphics[width=\linewidth]{register.png}

	\subsection{Counters}
	For each transition from \textbf{low to high}, the counter increments the binary output by 1.
	(Counting the transition from high to low does the same thing, but the lecture used rising-edge counters).

	\subsection{Decoders}
	Take in an $n$-bit number, and turn on one of $2^n$ outputs.

	\subsection{Multiplexer / Demultiplexer}
	Turns $n$ signals into a single signal, and back on the other end.

	\subsection{Fixed \& Programmable Logic}
	\textbf{Fixed logic circuits:} Pre-determined function.
	\textbf{Programmable logic:} FPGAs (reprogrammable, but still a significant cost to switching functions).
	\textbf{Stored program and re-programmable circuits:} Your computer right now.
	\section{Data Encoding}
	1 Byte = 8 bits.
	1 \textbf{Byte} encodes a character, integer, or pointer.
	1 \textbf{Word} is $n$ bytes, determined by the architecture.

	\subsection{Converting between bases}
	\textbf{Base 10 to Base N}
	Divide decimal \# by \# of new base.
	Take remainder as rightmost digit.
	Divide quotient of previous divide by new base.
	Repeat until quotient is zero.
	\textbf{Base N to Base 10}
	Take each column position of each digit, zero indexed, as $n$.
	For each column, do $c \cdot b^n$, where $c$ is the value of the column,
	and $b$ is the base value in base 10.

	\subsection{Signed Integers}
	\textbf{Sign-magnitude:} Dedicate one bit to the sign, and the rest to the magnitude.
	Range is $-(2^{n-1} - 1), +(2^{n-1} - 1)$,
	but has a positive and negative zero.
	\textbf{One's complement:} Invert all the bits to get negative number.
	Range is $-(2^{n-1} - 1), +(2^{n-1} - 1)$.
	\textbf{Two's complement:} Invert all bits, add a place value of 1 to get negative number.
	Eg, +6 is $0110$, so to get -6 go from $0110 \to 1001 \to 1010$.
	Range is $-(2^{n-1}), +(2^{n-1} - 1)$.

	\subsection{Types of architecture}
	\textbf{Von Neumann architecture:}
	\begin{itemize}
		\item Single memory block which contains both instructions and data.
		\item Offers complete flexibility: at any time, owner can change how much of the memory is devoted to programs and how much to data.
		\item Need to wait for the current instructions to be finished to load the next instructions \textbf{Von Neumann bottleneck}
		\item More popular.
	\end{itemize}
	\textbf{Harvard architecture:}
	\begin{itemize}
		\item  2 separate memory. One is used for instruction, one is used for data.
		\item inflexible, as u cannot use part of the instructional memory to store data and vise versa.
		\item no \textbf{Von Neumann bottleneck}
		\item Less popular. Sometimes used in small embedded systems.
	\end{itemize}
\end{multicols*}

\end{document}