\documentclass[letterpaper, 8pt]{extarticle}
\usepackage{amssymb,amsmath,amsthm,amsfonts}
\usepackage{multicol,multirow}
\usepackage{calc}
\usepackage{ifthen}
\usepackage[landscape]{geometry}
\usepackage[colorlinks=true,citecolor=blue,linkcolor=blue]{hyperref}

\usepackage{booktabs}
\usepackage{ulem}
\usepackage{enumitem}
\usepackage{tabulary}
\usepackage{graphicx}
\usepackage{siunitx}
\usepackage{tikz}
\usepackage{derivative}
\usepackage{svg}
\usepackage{listings}
\usepackage{setspace}
\usepackage{listings}
\usepackage{xcolor}
\usepackage{courier}
\usepackage{syntax}
\usepackage{mathpartir}
\usepackage{braket}

% minimal line spacing
% \setstretch{0.1}

% set borders (experimentally determined to minimize cutoff and maximize space on school printers)
\geometry{top=.25in,left=.25in,right=.25in,bottom=.35in}

% make figures work better in multicol
% \newenvironment{Figure}
% {\par\medskip\noindent\minipage}
% {\endminipage\par\medskip}

% \pagestyle{empty} % clear page

% rewrite section commands to be smaller
\makeatletter
\renewcommand{\section}{\@startsection{section}{1}{0mm}%
                                {-1explus -.5ex minus -.2ex}%
                                {0.5ex plus .2ex}%x
                                {\normalfont\normalsize\bfseries}}
\renewcommand{\subsection}{\@startsection{subsection}{2}{0mm}%
                                {-1explus -.5ex minus -.2ex}%
                                {0.5ex plus .2ex}%
                                {\normalfont\small\bfseries}}
\renewcommand{\subsubsection}{\@startsection{subsubsection}{3}{0mm}%
                                {-1ex plus -.5ex minus -.2ex}%
                                {1ex plus .2ex}%
                                {\normalfont\tiny\bfseries}}
\makeatother
\setcounter{secnumdepth}{0} % disable section numbering


% disable indenting
\setlength{\parindent}{0pt}
\setlength{\parskip}{0pt plus 0.5ex}

% Custom siunitx defs
\DeclareSIUnit\noop{\relax}
\NewDocumentCommand\prefixvalue{m}{%
\qty[prefix-mode=extract-exponent,print-unity-mantissa=false]{1}{#1\noop}
}

% Shorthand definitions
\newcommand{\To}{\Rightarrow}
\newcommand{\ttt}{\texttt}
\newcommand{\ra}{\rightarrow}

% condense itemize & enumerate
\let\olditemize=\itemize \let\endolditemize=\enditemize \renewenvironment{itemize}{\olditemize \itemsep0em}{\endolditemize}
\let\oldenumerate=\enumerate \let\endoldenumerate=\endenumerate \renewenvironment{enumerate}{\oldenumerate \itemsep0em}{\endoldenumerate}
\setlist[itemize]{noitemsep, topsep=0pt, leftmargin=*}
\setlist[enumerate]{noitemsep, topsep=0pt, leftmargin=*}

\title{3QI3}

\begin{document}
\raggedright
\tiny

% make listings look nicer
% \lstset{
%     tabsize = 2, %% set tab space width
%     showstringspaces = false, %% prevent space marking in strings, string is defined as the text that is generally printed directly to the console
%     basicstyle = \tiny\ttfamily, %% set listing font and size
%     breaklines = true, %% enable line breaking
%     numberstyle = \tiny,
%     postbreak = \mbox{\textcolor{red}{\(\hookrightarrow\)}\space}
% }

\begin{center}
    {\textbf{3QI3 - Popular Science Edition}} \\
\end{center}
% set column spacing rules
\setlength{\premulticols}{1pt}
\setlength{\postmulticols}{1pt}
\setlength{\multicolsep}{1pt}
\setlength{\columnsep}{2pt}
\begin{multicols*}{6}
    \section{Linalg}
    Matrix Multiplication:
    \(
    \begin{pmatrix}
        a & b \\
        c & d
    \end{pmatrix}
    \begin{pmatrix}
        e & f \\
        g & h
    \end{pmatrix}
    =
    \begin{pmatrix}
        ae+bg & af+bh \\
        ce+dg & cf+dh
    \end{pmatrix}
    \)

    \textbf{Adjoint (Hermitian Conjugate):}
    \(A^\dagger = A^*\)
    (transpose the matrix and take the complex conjugate of each element)

    \textbf{Complex Conjugate:}
    Flip the sign of the imaginary part of a complex number

    \section{Classical Information Theory}
    Shannon Entropy/Information: \(H = -\sum p(a_i) \log p(a_i)\)

    \section{Thermodynamics}
    Gibbs Entropy: \(S = -k \sum p_i \log p_i\)

    \section{Communication Theory}
    \textbf{Shannon's Noiseless Coding Theorem:}
    For a given message,
    we only need \(N H(p)\) bits to encode it,
    where \(H(p) = -\sum p_i \log p_i\)
    \textbf{Shannon's Noisy Coding Theorem:}
    On average, we need at least \(\frac{N_0}{1-H(q)}\)
    bits to encode one of \(2^{N_0}\) equally probable messages,
    where \(H(q) = -[q \log q + (1-q) \log (1-q)]\)

    \section{Dirac Notation}
    \(\bra{\Psi} \Longleftrightarrow \ket{\psi}^\dagger\)

    \begin{tabular}{@{}lc@{}}\toprule
        Ket                        & Matrix                                                      \\ \midrule
        \(\ket{0}\) or \(\ket{H}\) & \(\begin{bmatrix} 1 \\ 0 \end{bmatrix}\)                    \\
        \(\ket{1}\) or \(\ket{V}\) & \(\begin{bmatrix} 0 \\ 1 \end{bmatrix}\)                    \\
        Diagonal Up                & \(\frac{1}{\sqrt{2}}\begin{bmatrix} 1 \\ 1 \end{bmatrix}\)  \\
        Diagonal Down              & \(\frac{1}{\sqrt{2}}\begin{bmatrix} 1 \\ -1 \end{bmatrix}\) \\
        Left Circular              & \(\frac{1}{\sqrt{2}}\begin{bmatrix} 1 \\ i \end{bmatrix}\)  \\
        Right Circular             & \(\frac{1}{\sqrt{2}}\begin{bmatrix} 1 \\ -i \end{bmatrix}\) \\
        \bottomrule
    \end{tabular}

    \includegraphics[width=\linewidth]{Bloch\_sphere.svg.png}

    \(\ket{\Psi} = \cos\frac{\theta}{2}\ket{0}+e^{i \phi}\sin\frac{\theta}{2}\ket{1}\)

    \textbf{Change of basis}
    Let \(\theta\) be a rotation of basis vectors,
    counterclockwise.

    \(\ket{x} = \cos\theta\ket{x'} - \sin\theta\ket{y'}\)
    and
    \(\ket{y} = \sin\theta\ket{x'} + \cos\theta\ket{y'}\)

    where \(\ket{x'}\) and \(\ket{y'}\) are the new basis vectors.

    \textbf{Outer Product}
    Given that
    \(\ket{\psi} = \)
    \(\ket{\psi}\bra{\phi} = \begin{bmatrix} \psi_1\phi_1 & \psi_1\phi_2 \\ \psi_2\phi_1 & \psi_2\phi_2 \end{bmatrix}\)

    % TODO: quantum state tomography

\end{multicols*}

\end{document}
