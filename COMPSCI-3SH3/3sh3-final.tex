% IF YOU CAN SEE THIS GO CONTRIBUTE >:(

\documentclass[letterpaper, 8pt]{extarticle}
\usepackage{amssymb,amsmath,amsthm,amsfonts}
\usepackage{multicol,multirow}
\usepackage{calc}
\usepackage{ifthen}
\usepackage[landscape]{geometry}
\usepackage[colorlinks=true,citecolor=blue,linkcolor=blue]{hyperref}

\usepackage{booktabs}
\usepackage{ulem}
\usepackage{enumitem}
\usepackage{tabulary}
\usepackage{graphicx}
\usepackage{siunitx}
\usepackage{tikz}
\usepackage{derivative}
\usepackage{svg}
\usepackage{listings}
\usepackage{setspace}
\usepackage{listings}
\usepackage{xcolor}
\usepackage{courier}

% minimal line spacing
\setstretch{0.1}

% set borders (experimentally determined to minimize cutoff and maximize space on school printers)
\geometry{top=.25in,left=.25in,right=.25in,bottom=.35in}

% make figures work better in multicol
\newenvironment{Figure}
{\par\medskip\noindent\minipage}
{\endminipage\par\medskip}

\pagestyle{empty} % clear page

% rewrite section commands to be smaller
\makeatletter
\renewcommand{\section}{\@startsection{section}{1}{0mm}%
                                {-1explus -.5ex minus -.2ex}%
                                {0.5ex plus .2ex}%x
                                {\normalfont\normalsize\bfseries}}
\renewcommand{\subsection}{\@startsection{subsection}{2}{0mm}%
                                {-1explus -.5ex minus -.2ex}%
                                {0.5ex plus .2ex}%
                                {\normalfont\small\bfseries}}
\renewcommand{\subsubsection}{\@startsection{subsubsection}{3}{0mm}%
                                {-1ex plus -.5ex minus -.2ex}%
                                {1ex plus .2ex}%
                                {\normalfont\tiny\bfseries}}
\makeatother
\setcounter{secnumdepth}{0} % disable section numbering

% disable indenting
\setlength{\parindent}{0pt}
\setlength{\parskip}{0pt plus 0.5ex}

% Custom siunitx defs
\DeclareSIUnit\noop{\relax}
\NewDocumentCommand\prefixvalue{m}{%
\qty[prefix-mode=extract-exponent,print-unity-mantissa=false]{1}{#1\noop}
}

% Shorthand definitions
\newcommand{\To}{\Rightarrow}
\newcommand{\ttt}{\texttt}

% condense itemize & enumerate
\let\olditemize=\itemize \let\endolditemize=\enditemize \renewenvironment{itemize}{\olditemize \itemsep0em}{\endolditemize}
\let\oldenumerate=\enumerate \let\endoldenumerate=\endenumerate \renewenvironment{enumerate}{\oldenumerate \itemsep0em}{\endoldenumerate}
\setlist[itemize]{noitemsep, topsep=0pt, leftmargin=*}
\setlist[enumerate]{noitemsep, topsep=0pt, leftmargin=*}

\title{3SH3}

\begin{document}
\raggedright
\tiny

% make listings look nicer
\lstset{
    tabsize = 2, %% set tab space width
    showstringspaces = false, %% prevent space marking in strings, string is defined as the text that is generally printed directly to the console
    basicstyle = \tiny\ttfamily, %% set listing font and size
    breaklines = true, %% enable line breaking
    numberstyle = \tiny,
    postbreak = \mbox{\textcolor{red}{\(\hookrightarrow\)}\space}
}

\begin{center}
    {\textbf{3SH3 Final -- Linus Torvalds Edition}} \\
\end{center}
% set column spacing rules
\setlength{\premulticols}{1pt}
\setlength{\postmulticols}{1pt}
\setlength{\multicolsep}{1pt}
\setlength{\columnsep}{2pt}
\begin{multicols*}{6}
    % sections based on slide titles
    \section{Overview}
    \section{Processes}
    \section{Threads}
        \subsection{Amdahl's Law}
            $speedup \leq \frac{1}{S+(1-S)/N}$ where $S$ is the serial 
            portion of the application and $N$ is the number of processing cores
    \section{Synchronization}
        \subsection{Semaphores}
            \texttt{wait()} = \texttt{P()}, \texttt{signal()} = \texttt{V()},
            \texttt{wait()} decrements the semaphore, \texttt{signal()} increments the 
            semaphore.
    \section{Deadlocks}
    \section{Scheduling}
        \subsection{Predicting next CPU burst Formula}
            $\tau_{n+1} = \alpha \cdot t_n + (1-\alpha) \cdot \tau_n$
            where $t_n$ is the value of the nth CPU burst, $0 \leq \alpha \leq 1$
        \subsection{Scheduling Algorithms Formulas}
            \textbf{Turnaround time of a process} is difference between when 
            the process finishes execution and its arrival time; 
            turnaround time = process finish time - start time.
            \textbf{Waiting time of a process} is how long a process does not execute 
            on the CPU from its arrival.
            \textbf{CPU utilization rate} is the time the cpu spends executing 
            processes divided by the total time (time spent executing + time spent idle)
    \section{Memory Management}
    \section{Virtual Memory}
    \section{File System}
    \section{Mass Storage Systems}
    \section{Conversions}
        1 second = 1000 milliseconds, 1 millisecond = 1000 microseconds, 
        1 microsecond = 1000 nanoseconds
\end{multicols*}

\end{document}
